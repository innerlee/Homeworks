\begin{abstract}
We view the task of Generative Adversarial Networks as manifold learning.
Instead calling it as noise,
we see the latent space as the coordinate of the data manifold.
The generator is the function that maps coordinates to data manifold.
Thus, other than the traditional approach that investigating the
probabilistic properties of the noise distribution and the data distribution,
we ask whether the geometric and topological properties of latent space and
data manifold interact with each other.
Specifically, we visualize the effect of dimensionality,
connectivity, and topology of latent space and data manifold
using specially designed toy experiments.
\end{abstract}
