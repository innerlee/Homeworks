\clearpage
\section{Note B}
Reading~\cite{raskutti2010restricted}.

\subsection{Prelude}

\subsubsection{Recall}

We are interested in establishing the following \emph{restricted eigenvalue} (RE) condition:

\begin{equation}
    \frac{\|X\Delta\|_2^2}{2n}\ge \kappa\|\Delta\|_2^2\quad\forall\Delta\in\cC=
        \{\Delta\in\real^d:\|\Delta_{S^c}\|_1\le3\|\Delta_S\|_1\},
\end{equation}
where $\kappa>0$, $X\in\real^{n\times d}$.

In general,
one would not expect it to hold for arbitrary $X$;
\eg fix $\Delta\in\cC$ and construct $X$
\st $\Delta\in\nullsp(X)$.
Thus some assumptions on $X$ are needed.

Let us assume that the rows of $X$ are iid $\cN(0,I)$.
Our goal is to prove

\begin{thm}
    With prob. $\ge 1-c_1\exp(-c_2 n)$,
    for some $c_1,c_2>0$.
    \begin{equation}
        \frac{\|X\Delta\|_2}{\sqrt{n}}\ge
            \frac{1}{4}\|\Delta\|_2-9\sqrt{\frac{\log d}{n}}\|\Delta\|_1
            \quad \forall \Delta\in\real^d.
    \end{equation}
\end{thm}

Note that the theorem trivially holds for $\Delta=0$.
Thus without loss of generality,
we may assume $\|\Delta\|_2=1$.
The proof of the theorem consists of three main steps:

\paragraph{Step 1.}
For a given $r>0$, define
\begin{equation}
    V(r)=\{\Delta\in\real^d:\|\Delta\|_2=1,\|\Delta\|_1\le r\}.
\end{equation}
\begin{pro}\label{pro:b1}
    \begin{equation}
        \Ebb\bigg[\inf_{\Delta\in V(r)}\frac{\|X\Delta\|_2}{\sqrt{n}}\bigg]\ge
            3\bigg[\frac{1}{4}-\sqrt{\frac{\log d}{n}}r\bigg],
    \end{equation}
    whenever $V(r)\neq \emptyset$.
\end{pro}
\begin{proof}
The quantity of interest here is
\begin{equation}
    \widetilde Q(r,X) = \inf_{\Delta\in V(r)}\|X\Delta\|_2
        = \adjustlimits\inf_{\Delta\in V(r)}\sup_{u\in S^{n-1}}u^TX\Delta.
\end{equation}
Note that for each $(u,\Delta)\in S^{n-1}\times V(r)$,
\begin{equation}
Y_{u,\Delta}=u^TX\Delta
\end{equation}
is a zero mean Gaussian RV.
To get a lower bound on $\Ebb[\widetilde Q(r,X)]$,
or equivalently,
an upper bound on
\begin{equation}
    \Ebb[-\widetilde Q(r,X)] =
        \Ebb\bigg[\adjustlimits\sup_{\Delta\in V(r)}\inf_{u\in S^{n-1}}u^TX\Delta\bigg],
\end{equation}
a powerful idea is to construct another family of Gaussian random variables
$\{Z_{u,\Delta}\}$,
such that $\Ebb[\adjustlimits\sup_{\Delta\in V(r)}\inf_{u\in S^{n-1}}Z_{u,\Delta}]$
is easy to compute and is related to
$\Ebb[\adjustlimits\sup_{\Delta\in V(r)}\inf_{u\in S^{n-1}}Y_{u,\Delta}]$.
This is the content of Gordon's inequality.
\begin{fact}[Gordon's Inequality]\label{fact:gordon}
Let $U,V$ be arbitrary index sets.
Consider two families $\{Y_{u,v}\}$ and $\{Z_{u,v}\}$ of zero-mean Gaussian RVs.
Suppose that
\begin{equation}
    \sigma(Y_{u,v}-Y_{u',v'})\le\sigma(Z_{u,v}-Z_{u',v'})\quad
    \forall(u,v),(u',v')\in U\times V,
\end{equation}
and
\begin{equation}
    \sigma(Y_{u,v}-Y_{u,v'})\le\sigma(Z_{u,v}-Z_{u,v'})\quad
    \forall u\in U,~ v,v'\in V.
\end{equation}
Then,
\begin{equation}
    \Ebb\bigg[\adjustlimits\sup_{u\in U}\inf_{v\in V}Y_{u,v}\bigg]\le
    \Ebb\bigg[\adjustlimits\sup_{u\in U}\inf_{v\in V}Z_{u,v}\bigg].
\end{equation}
\end{fact}

To apply Gordon's inequality,
let us first compute $\sigma(Y_{u,\Delta}-Y_{u',\Delta'})$,
and see what properties are needed for $Z_{u,\Delta}$.
By definition,

\end{proof}

\paragraph{Step 2.}
We show that the random variable $Q(r,X)$ is concentrated around its mean,
where
\begin{equation}
    Q(r,X)=\inf_{\Delta\in V(r)}\frac{\|X\Delta\|_2}{\sqrt{n}}.
\end{equation}
\begin{pro}\label{pro:b2}
Let $r>0$ \st $V(r)\ne \emptyset$.
Then
\begin{equation}
    \Pr\big[\;|Q(r,X)-\Ebb[Q(r,X)]|\ge\frac{1}{2}t(r)\;\big]\le 2\exp(-nt^2(r)/8),
\end{equation}
\end{pro}
where
\begin{equation}
    t(r)=\frac{1}{4}+3\sqrt{\frac{\log d}{n}}r.
\end{equation}

\paragraph{Step 3.}
The results of Proposition~\ref{pro:b1} and~\ref{pro:b2} show that
with probability at least $1-2\exp(-nt^2(r)/8)$,
\begin{equation}
\begin{split}
    Q(r,X) &= \inf_{\Delta\in V(r)}\frac{\|X\Delta\|_2}{\sqrt{n}}
        \ge \Ebb[Q(r,X)]-\frac{1}{2}t(r) \\
        &\ge 1-\frac{3}{2}t(r)=\frac{5}{8}-\frac{9}{2}\sqrt{\frac{\log d}{n}}r.
\end{split}
\end{equation}
Here,
we need $\|\Delta\|_1\le r$,
where $r$ is fixed.
However,
we need the above to hold for all $r$.
This is the goal of this step.

\subsubsection{Probability Tools Used}

\begin{itemize}
    \item Comparison inequality for Gaussian process.
    \item Concentration of measure for Lipschitz functions of Gaussians.
    \item ``Peeling argument'' from empirical process theory.
\end{itemize}
