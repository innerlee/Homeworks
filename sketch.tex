\section{Sketching}

Let $n=20000$ and $d=21$ as in our dataset.
The general framework contains three steps:
\begin{enumerate}
    \item L1-norm subspace embedding.
        This step can use \emph{Cauchy} or \emph{Exponential} mehtod.
        The output is a probability vector $p$.
        There are two parameters in this step.
        The first is the number of rows in \emph{Cauchy} and \emph{Exponential}
        subspace embedding.
        Some guidance on chosing this number are
        $\mathcal{O}(d\log(d))$ as in~\cite[Theorem 36]{woodruff2014sketching}
        and $d \cdot \text{poly}(\log(d))$
        as in~\cite[Theorem 41]{woodruff2014sketching}.
        We fix this number to $21\times \log(21)\approx64$.
        The other is the columns of Gaussian sketching which is used for
        accelerating algorithm.
        We fix it to be 16.
    \item Sampling.
        This step is similar like the Leverage score sketching.
        After this step, we get a shirinked sized problem.
        The number in sampling is controlled by parameter $r$.
        We tested $r=100$ and $r=500$ for reference.
    \item Solving small sized problem.
        We use LP solver to get the solution.
\end{enumerate}
Testing results are shown in Table~\ref{tab:grand}.

\begin{table}[htb]
  \setlength{\tabcolsep}{2.6pt}
  \caption{The performances of different Sketching techniques.
  $r$ is the parameter in step~2.
  time-$i$ is the average time spent on step~$i$.
    }
  \label{tab:grand}
  \centering
  {\small
  \begin{tabular}{lllllllllllll}
    \toprule
    algorithm & repeat & $r$ & time1 & time2 & time3 & time & min loss & max loss & median loss & mean loss & std loss & rel err \\
    \midrule
    Baseline & 1 & - & - & - & - & 97.30 & 3.428 & 3.428 & 3.428 & 3.428 & - & - \\
    Cauchy & 100 & 100 & 0.129 & 0.014 & 0.010 & 0.154 & 3.5507 & 4.7928 & 3.9607 & 3.9855 & 0.2484 & 3.56\% \\
    Cauchy & 100 & 500 & 0.131 & 0.014 & 0.125 & 0.272 & 3.4497 & 3.5875 & 3.5023 & 3.5045 & 0.0259 & 0.62\% \\
    Exponential & 100 & 100 & 0.012 &  0.002 & 0.011 & 0.025 & 3.6121 & 4.6369 & 3.8575 & 3.9185 & 0.2231 & 5.35\% \\
    Exponential & 100 & 500 & 0.014 & 0.002 & 0.129 & 0.146 & 3.4486 & 3.5924 & 3.4934 & 3.4971 & 0.0251 & 0.58\% \\
    \bottomrule
  \end{tabular}
  }
\end{table}

%
\subsection{Cauchy}
The minimum point when $r=500$ is
{
\def\OldComma{,}
\catcode`\,=13
\def,{%
	\ifmmode%
	\OldComma\discretionary{}{}{}%
	\else%
	\OldComma%
	\fi%
}%
$x^* = (0.657501,0.0776499,-0.133653,-0.0633284,-0.876821,-0.759774,1.41801,0.980646,0.0688024,-0.0893941,-0.221192,0.0627549,0.0966104,0.0311747,0.359389,0.235925,-0.00986104,0.144246,-0.0530562,-0.0743144,-0.19942)$
with loss $f(x^*) = 3.4497$.
}

%
\subsection{Exponential}
The minimum point when $r=500$ is
{
\def\OldComma{,}
\catcode`\,=13
\def,{%
	\ifmmode%
	\OldComma\discretionary{}{}{}%
	\else%
	\OldComma%
	\fi%
}%
$x^* = (0.493382,0.42278,0.242891,-0.476517,-0.753511,-0.43696,0.680251,0.761099,0.196092,-0.137095,-0.35945,0.181037,-0.033585,-0.037471,0.506696,0.282556,-0.221777,0.00231164,0.0114419,-0.144334,-0.0618985,-1.0)$
with loss $f(x^*) = 3.4486$.
}
