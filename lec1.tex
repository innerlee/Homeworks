\section{Arnold Conjecture}

%%%%%%%%%%%%%%%%%%%%%%%%%%%%%%%%%%%%%%%%%%%%%%%%%%%%%%%%%%%%%%%%%%%%%%%%%%%%%%%%
\subsection{Arnold Conjecture (A)} %1.1

$(M^{2n}, \omega)$, $M$ closed, $\omega$ \textbf{symplectic}.

\begin{framed}
    A symplectic form is a $2$-form satisfying an algebraic condition --
    nondegeneracy -- and an analytical condition -- closedness.
    \ie, $d\omega=0$, and $\omega^n$ is a volume form.
\end{framed}

Let $\phi$ be a symplectomorphism of $(M, \omega)$.
$H:M\to\real$ be a function.
Then there exists vector field $V_H$,
called \textbf{Hamiltonian vector field}, \st $dH=\omega(V_H,\cdot)$.
Let $\phi_H$ be the flow of $V_H$,

\begin{framed}
    Say $M$ compact, then $\forall V\in\Gamma(M,T_M)$,
    $\exists$ \textbf{flow} $\phi:M\times\real\to M$, where
    $f_t=\phi(\cdot, t)\in \Diff(M)$, \st
    $f_0=\id_M, \frac{df_t}{dt}\big|_{t=0}=V, f_{t_1+t_2}=f_{t_1}\circ f_{t_2}$.
\end{framed}

then $f_t=\phi_H(\cdot, t)$ is a symplectomorphism, $\forall t$.

\begin{framed}
    \cite[P105]{da2001lectures}
    We have $\frac{d}{dt}f_t^*(\omega) = f_t^*\cL_{V_H}\omega
    =f_t^*(d\iota_{V_H}\omega+\iota_{V_H}d\omega) = 0$.
\end{framed}

A vector field $V$ on $M$ preserving $\omega$ is called \textbf{symplectic vector field}.
It is equivalent to the condition that $\omega(V, \cdot)$ is closed.
However, it is not necessarily exact.

Let $H_t:M\to\real$ be a $1$-parameter family of Hamiltonian function, $t\in[0,1]$.
$\frac{d\phi_{H_t}}{dt}:=V_{H_t}$.
Define $\phi_H:=\phi_{H_1}$.

\begin{conj}[A]
    $(M^{2n}, \omega)$ is symplectic. $M$ is closed.
    $\phi_H$ is non-degenerately defined by $H_t$.
    Suppose all fixed points of $\phi_H$ is non-degenerate, then
    \begin{equation}
        |\Fix~\phi_H | \ge \sum_i b_i(M;\real).
    \end{equation}
\end{conj}

Compare with a general diffeomorphism $f:M\to M$.
Suppose all fixed points are non-degenerate.
By Lefschetz fixed point theorem,

\begin{framed}
    \missref
\end{framed}

\begin{equation}
    \#\Fix(f) = \sum(-1)^i b_i(M).
\end{equation}

\textbf{Remark}.
\warn
One can find $f$ that achieves the lower bound.
The lower bound fails in general for non-Hamiltonian symplectomorphisms.

\textbf{Example}

$(T^2,\omega)$, $\phi$ is a symplectomorphism.
If $\phi$ is Hamiltonian, then $\phi$ has at least $2$ fixed points.
\eg,
take $H$ to be Morse function,
then fixed points of $H$ are two critical points of $H$.

\todo (need drawing)
Let $\omega=dx\wedge dy$.
$\phi$ is not Hamiltonian.
$\omega(\cdot, V)=dy$ is a non-exact closed form.


\begin{conj}[Generalized Arnold Conjecture]
    $\phi$: symplectomorphism that is isotopic to $\Id$.
    \begin{equation}
        |\Fix~\phi | \ge \sum_i b_i(M;\Lambda)
    \end{equation}
    where $\Lambda$ is a certain local coefficient depending on the
    cohomology class of $\omega(V,\cdot)$.
\end{conj}

\textbf{Remark}. Complicated local coefficient tends to lower $b_i$.

\subsubsection{Lagrangian}
