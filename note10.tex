\clearpage
\section{Note J}

\subsection{Prelude}

\subsubsection{Recall}

In the past lectures,
we focused on regularized loss minimization problems that have
non-convex loss functions and/or non-convex regularizations.
By stipulating that the loss function satisfies RSC and that
the regularizers is not ``too non-convex'',
we essentially turned the problem into a convex one,
for which standard techniques apply.

\subsubsection{This Lecture}

In this lecture,
we depart from the regularized loss minimization problem and
consider another quite different estimation problem:
the \emph{phase synchronization problem}.
In this problem,
one is interested in recovering a collection of phases $\{e^{i\theta_k}\}$
based on noisy measurements of relative phases $\{e^{i(\theta_j-\theta_k)}\}$.
Specifically,
let
\begin{equation}
    z^*\in T^n\triangleq\{w\in\cpx^n:|w_1|=\cdots=|w_n|=1\}
\end{equation}
be the unknown phase vector we wish to recover.
We have noisy measurements of the form
\begin{equation}\label{eq:probj}
    C_{jl}=z_j^*\bar z_l^*+\Delta_{jl},\quad 1\le j<l\le n,
\end{equation}
where $\Delta_{jk}$ is the noise associated with the $(j,k)$-th measurement.
To recover $z^*$,
it is natural to formulate the following least squares problem,
\begin{equation}
    \hat z \in\argmin_{z\in T^n}\sum_{1\le j<l\le n}|c_{jl}-z_j\bar z_l|^2.
\end{equation}
Using the fact that $|z_j\bar z_l|^2=1$ for any $z\in T^n$,
the above is equivalent to
\begin{equation}\label{eq:j_p}
    \hat z \in\argmax_{z\in T^n}z^HCz,
\end{equation}
where $C_{jj}=1$ for $j=1,\dots,n$.

Note that neither the objective function nor the constraints set of
\eqref{eq:j_p} is convex.
Moreover,
there will be multiple optimal solutions to \eqref{eq:j_p},
because whenever $\hat z$ is optimal,
so is $e^{i\theta}\hat z$ for any $\theta\in[0,2\pi)$.
This also suggests that we can only recover $z^*$ up to a global phase.
Hence,
we define the following distance metric to measure the closeness of
any $z\in T^n$ to the target vector $z^*$,
\begin{equation}
    d_2(z,z^*)=\min_{\theta\in[0,2\pi)}\|z-e^{i\theta}z^*\|_2.
\end{equation}

Following our earlier studies,
a natural first question is to study the estimation performance of $\hat z$
\wrt the metric $d_2$.
As it turns out,
this is rather straightforward.
To begin,
observe from \eqref{eq:probj} that
\begin{equation}
    C=z^*(z^*)^H+\Delta,
\end{equation}
where $\Delta$ is a Hermitian matrix whose diagonal is zero
and its above-diagonal entries are given by $\Delta_{jl}$.
Then,
we have the following,
\begin{pro}\label{pro:j1}
Let $z\in\cpx^n$ be such that $\|z\|_2^2=n$ and $(z^*)^HCz^*\le z^HCz$
(in particular,
these conditions are satisfied by an optimal solution
$\hat z$ to \eqref{eq:j_p}).
Then,
\begin{equation}
    d_2(z,z^*) = \sqrt{2\big(n-|z^Hz^*|\big)}\le\frac{4\|\Delta\|}{\sqrt{n}}.
\end{equation}
\end{pro}
\begin{proof}
By definition,
\begin{equation}
\begin{split}
    d_2(z,z^*)^2 &=\min_{\theta\in[0,2\pi)}\|z-e^{i\theta}z^*\|_2^2 \\
        &= 2\bigg(n-\max_{\theta\in[0,2\pi)}\re(e^{i\theta}z^Hz^*)\bigg) \\
        &= 2(n-|z^Hz^*|).
\end{split}
\end{equation}
Now,
without loss,
suppose that $z^Hz^*=|z^Hz^*|$.
By assumption,
\begin{equation}
    z^HCz=|z^Hz^*|^2+z^H\Delta z\ge(z^*)^HCz^*=n^2+(z^*)^H\Delta z^*.
\end{equation}
This gives
\begin{equation}
    n^2-|z^Hz^*|^2\le z^H\Delta z-(z^*)^H\Delta z^*.
\end{equation}
Dividing both sides by $n$ and observing that $|z^Hz^*|\le n$ and
\begin{equation}
    n^2-|z^Hz^*|^2=(n-|z^Hz^*|)(n+|z^Hz^*|),
\end{equation}
we have
\begin{equation}
\begin{split}
    n-|z^Hz^*|
        &\le \frac{1}{n}\big(z^H\Delta z-(z^*)^H\Delta z^*\big) \\
        &= \frac{1}{n}\re\big((z-z^*)^H\Delta (z+z^*)\big) \\
        &\le \frac{1}{n}\|\Delta\|~\|z-z^*\|_2~\|z+z^*\|_2 \\
        &\le \frac{2}{\sqrt{n}}\|\Delta\|~\|z-z^*\|_2
\end{split}
\end{equation}
Note that $d_2(z,z^*)=\|z-z^*\|_2$,
it follows that
\begin{equation}
    d_2(z,z^*)\le \frac{4\|\Delta\|}{\sqrt{n}}.
\end{equation}
\end{proof}

In view of Proposition~\ref{pro:j1},
it is natural to ask if we can find an optimal solution $\hat z$
to \eqref{eq:j_p} efficiently.
Noting that projection onto the non-convex set $T^n$ is efficiently computable,
let us consider the following simple projected gradient scheme,
\begin{equation}
\begin{split}
    w^k &\leftarrow z^k + \frac{\alpha_k}{n}Cz^k, \\
    z^{k+1}&\leftarrow \frac{w^k}{|w^k|},
\end{split}
\end{equation}
where $\alpha_k>0$ is the step size,
and for a vector $w\in\cpx^n$,
\todo
