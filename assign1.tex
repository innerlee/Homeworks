\section{Assignment 1}

%%%%%%%%%%%%%%%%%%%%%%%%%%%%%%%%%%%%%%%%%%%%%%%%%%%%%%%%%%%%%%%%%%%%%%%%%%%%%%%%
\subsection{} %1.1

Since
\begin{align}
    P(Y_i=0) &= P(X_i\le\mu) = F_X(\mu), \\
    P(Y_i=1) &= 1 - F_X(\mu).
\end{align}
We know $Y_i\sim Bernoulli(1-F_X(\mu))$ and are iid.
So the sum is a Binomial distribution. \ie
\begin{equation}
    \sum_{i=1}^n Y_i = Binomial(n, 1 - F_X(\mu)).
\end{equation}


%%%%%%%%%%%%%%%%%%%%%%%%%%%%%%%%%%%%%%%%%%%%%%%%%%%%%%%%%%%%%%%%%%%%%%%%%%%%%%%%
\subsection{} %1.2

If we model the Binomial distribution as counting heads of tossing a biased coin,
then the sum of independent Binomial distributions is just tossing more times.
So the result should be another Binomial distribution.
\ie, $\sum_{i=1}^{k}X_i \sim Binomial\big(\sum_i n_i, p\big)$.

Formally, we only need to prove for the case of $k=2$.
\begin{align}
    P(X_1+X_2=t) &= \sum_{i=0}^t P(X_1=i)P(X_2=t-i) \\
                 &= \sum_{i=0}^t \binom{n_1}{i}p^{i}(1-p)^{n_1-i}\binom{n_2}{t-i}p^{t-i}(1-p)^{n_2-(t-i)} \\
                 &= p^{t}(1-p)^{n_1+n_2-t}\sum_{i=0}^t \binom{n_1}{i}\binom{n_2}{t-i} \\
                 &= \binom{n_1+n_2}{t}p^{t}(1-p)^{n_1+n_2-t}
\end{align}
So $X_1+X_2\sim Binomial(n_1+n_2, p)$.
Simple induction will suffice for proving the general case for $k>2$.


%%%%%%%%%%%%%%%%%%%%%%%%%%%%%%%%%%%%%%%%%%%%%%%%%%%%%%%%%%%%%%%%%%%%%%%%%%%%%%%%
\subsection{} %1.3
