\section{Problem 2}~\label{sec:prob2}

\subsection{} % 2.1

\begin{equation}
\begin{split}
    \vh_t
        &= F_\theta(\vh_{t-1},\vx_t) \\
        &= F_\theta(F_\theta(\vh_{t-2},\vx_{t-1}),\vx_t) \\
        &= F_\theta(F_\theta(F_\theta(\vh_{t-3},\vx_{t-2}),\vx_{t-1}),\vx_t) \\
        &= F_\theta(F_\theta(F_\theta(F_\theta(\vh_{t-4},\vx_{t-3}),\vx_{t-2}),\vx_{t-1}),\vx_t) \\
        &= \cdots\cdots (\text{unfold until reach } \vx_1)\\
\end{split}
\end{equation}

\subsection{} % 2.2

(1) RNN shares parameter.
This enables handling sequences of different length
and improves the generalization power of the model at the same time.

(2) RNN compresses the information of whole history into one hidden state variable.
This avoids the exponential growth of model complexity.
