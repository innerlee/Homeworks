\section{Problem 3}

Since $h(\cA(X))$ is smooth,
we compute the derivative
\begin{equation}
    D:=\frac{\partial h(\cA(X))}{\partial X} =
    \begin{bmatrix}
        \frac{3}{2}X_{11}-2X_{22}-\frac{5}{2} & 0 \\
        0   & -2X_{11}+3X_{22}+1 \\
    \end{bmatrix}.
\end{equation}
The two numbers in the matrix come from
\begin{equation}
\begin{split}
    \frac{\partial h(y)}{\partial y}
        &= (B^{\frac{1}{2}})^T(B^{\frac{1}{2}}y-B^{-\frac{1}{2}}d) \\
        &= By-d.
\end{split}
\end{equation}

The optimality condition at point $X$ is
\begin{equation}
    0\in\{D+W:W\in\partial\|X\|_*\}.
\end{equation}
Or equivalently,
\begin{equation}\label{eq:cond}
    -D\in\partial\|X\|_*.
\end{equation}

For $X=0$,
$\partial\|X\|_*=\{W:\|W\|\le1\}$.
However,
$\|-D\|=2.5>1$.
So $X=0$ is not a solution.

If $X$ is rank 2,
suppose one of its SVD is $X=U\Sigma V^T$
and $\sigma_1\ge\sigma_2>0$.
Then $\partial\|X\|_*=UV^T$.
We get the constraint
\begin{equation}
    -D=UV^T.
\end{equation}
The right hand side can be seen as an SVD of $-D$
with $\sigma_1(D)=\sigma_2(D)=1$.
Since $D$ is a diagonal matrix,
we have
\begin{equation}
\begin{split}
    \frac{3}{2}X_{11}-2X_{22}-\frac{5}{2} &= \pm1, \\
    -2X_{11}+3X_{22}+1 &= \pm1.
\end{split}
\end{equation}
