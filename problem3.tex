\section{Problem 3}

Since $h(\cA(X))$ is smooth,
we compute the derivative
\begin{equation}
    D:=\frac{\partial h(\cA(X))}{\partial X} =
    \begin{bmatrix}
        \frac{3}{2}X_{11}-2X_{22}-\frac{5}{2} & 0 \\
        0   & -2X_{11}+3X_{22}+1 \\
    \end{bmatrix}.
\end{equation}
The two numbers in the matrix come from
\begin{equation}
\begin{split}
    \frac{\partial h(y)}{\partial y}
        &= (B^{\frac{1}{2}})^T(B^{\frac{1}{2}}y-B^{-\frac{1}{2}}d) \\
        &= By-d.
\end{split}
\end{equation}

The optimality condition at point $X$ is
\begin{equation}
    0\in\{D+W:W\in\partial\|X\|_*\}.
\end{equation}
Or equivalently,
\begin{equation}\label{eq:cond}
    -D\in\partial\|X\|_*.
\end{equation}

\textbf{For $X=0$ being rank 0},
$\partial\|X\|_*=\{W:\|W\|\le1\}$.
However,
$\|-D\|=2.5>1$.
Hence $-D\notin\partial\|X\|_*$ and
we know $X=0$ is not a optimal solution.

\textbf{If $X$ is rank 1},
Then an SVD of $X$ can be written as
\begin{equation}
    X=[u\ u^\perp] \diag(\sigma,0) [v\ v^\perp]^T = \sigma uv^T,
\end{equation}
where $\sigma>0, u=(u_1,u_2), u^\perp=(-u_2,u_1), v=(v_1,v_2), v^\perp=(-v_2,v_1)$.
So
\begin{equation}
\begin{split}
    \partial\|X\|_*
        &=\bigg\{[u\ u^\perp]
            \begin{bmatrix}
                1 & 0    \\
                0 & w \\
            \end{bmatrix}
            [v\ v^\perp]^T:|w|\le1\bigg\} \\
        &=\bigg\{
            \begin{bmatrix}
                u_1v_1+wu_2v_2 & u_1v_2-wu_2v_1 \\
                u_2v_1-wu_1v_2 & u_1v_1+wu_2v_2 \\
            \end{bmatrix}
            :|w|\le1\bigg\}
\end{split}
\end{equation}
So, the optimality condition becomes $\exists |w|\le1$, \st
\begin{equation}
    -D=\begin{bmatrix}
            u_1v_1+wu_2v_2 & u_1v_2-wu_2v_1 \\
            u_2v_1-wu_1v_2 & u_1v_1+wu_2v_2 \\
        \end{bmatrix}.
\end{equation}
We have
\begin{equation}
u_1v_2=wu_2v_1,\quad u_2v_1=wu_1v_2.
\end{equation}

\textbf{If} $u_1v_2=u_2v_1=0$,
then must have $u=v=(1,0)$ or $u=v=(0,1)$.

If $u=v=(1,0)$,
then $-D=I$ and $X=\diag(\sigma,0)$.
We get $X_{11}=1,X_{22}=0$.
So
\begin{equation}
    \hat X=\diag(1,0) \text{ satisfies the optimality condition}.
\end{equation}
Compute the objective value and we have
$h(\cA(\hat X))+\|\hat X\|_*=4$.

If $u=v=(0,1)$,
then $-D=wI$ and $X=\diag(0,\sigma)$.
Solve them we get $\sigma=0.7, w=1.1>1$,
a contradiction with condition $|w|\le1$.

\textbf{If} $u_1v_2\ne0$,
then we have $u_1v_2=wu_2v_1=w^2u_1v_2$
and thus $w=\pm1$.

If $w=1$,
we have $u_1v_2-u_2v_1=0$,
\ie $u$ and $v$ are colinear.
We must have $u=\pm v$.
If $u=v$,
then $X=\sigma uu^T$ and $-D=I$.
we get $X_{11}=1, X_{22}=0$.
Then we have $\sigma u_2^2=X_{22}=0$
and thus $u_2=0$,
a contradiction with $u_2v_1=u_1v_2\ne0$.
If $u=-v$,
then $X=-\sigma uu^T$ and $-D=-I$.
we get $X_{11}=21, X_{22}=14$.
Then we have $\|X\|_*\ge21$,
a value being too large for an optimal.

If $w=-1$,
we have $u_1v_2+u_2v_1=0$,
and thus $(u_1,u_2)$ and $(-v_1,v_2)$ are colinear.
We must have $(u_1,u_2)=\pm(-v_1,v_2)$.
If $(u_1,u_2)=(-v_1,v_2)$,
then $-D=-I$ and we have $X_{11}=21, X_{22}=14$.
This is too large.
If $(u_1,u_2)=(v_1,-v_2)$,
then $X=\sigma uv^T$ and $-D=I$.
we get $X_{11}=1, X_{22}=0$.
Then we have $-\sigma u_2^2=X_{22}=0$
and thus $u_2=0$,
a contradiction with $u_2v_1=u_1v_2\ne0$.

\textbf{If $X$ is rank 2},
suppose one of its SVD is $X=U\Sigma V^T$
and $\sigma_1\ge\sigma_2>0$.
Then $\partial\|X\|_*=UV^T$.
We get the constraint
\begin{equation}
    -D=UV^T.
\end{equation}
The right hand side can be seen as an SVD of $-D$
with $\sigma_1(D)=\sigma_2(D)=1$.
Since $D$ is a diagonal matrix,
we have
\begin{equation}
    D=
    \begin{bmatrix}
        \pm1 & 0    \\
        0    & \pm1 \\
    \end{bmatrix}.
\end{equation}

If $D=-I$,
then we get $X_{11}=1, X_{22}=0$.
So $\sigma_1(X)\ge1$.
Since $X$ is rank 2,
$\sigma_2(X)>0$.
We have $\|X\|_*>1$.
So the objective function will have value larger than $4$.

If $D=I$,
then we get $X_{11}=21, X_{22}=14$.
This is too large.

If $D=
    \begin{bmatrix}
        -1 & 0 \\
        0  & 1 \\
    \end{bmatrix}$,
then we get $X_{11}=9, X_{22}=6$.
This is also too large.

If $D=
    \begin{bmatrix}
        1 & 0  \\
        0 & -1 \\
    \end{bmatrix}$,
then we get $X_{11}=-7/5, X_{22}=-8/5$.
Let $U=[u\ k_1u^\perp],V=[v\ k_2v^\perp]$,
where $k_1=\pm1,k_2=\pm1$.
Then $-D=UV^T=\begin{bmatrix}
        u_1v_1+k_1k_2u_2v_2 & u_1v_2-k_1k_2u_2v_1 \\
        u_2v_1-k_1k_2u_1v_2 & u_1v_1+k_1k_2u_2v_2 \\
    \end{bmatrix}$.
This leads to a contradiction
because the diagonal elements of the right hand side have the same sign,
while those of the left hand side matrix $D$ have opposite ones.

As argued above,
we found that the unique optimal solution is $\hat X=\diag(1,0)$.
