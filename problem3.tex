\section{Problem 3}

First observe that the right hand side of the condition~(b)
is $1$-homogeneous.
This motivate us to consider if we can reduce the problem to
the case when $\|\Delta\|_2=1$,
since the absolute length is unimportant here.
However,
we need to deal with the left hand side
because $\Delta$ appears in the term $\nabla\cL(\theta^*+\Delta)$.
This is easy to handle due to the convexity of $\cL$.
Since $(\nabla\cL(\theta^*+\Delta))^T\Delta/\|\Delta\|_2$ is
the directional derivative of $\cL$ at $(\theta^*+\Delta)$
along direction $\Delta$.
Restrict $\cL$ on line $t\mapsto\theta^*+t\Delta$,
function $f(t):=\cL(\theta^*+t\Delta)$ is convex
and its derivative is increasing.
Thus for $\|\Delta\|_2\ge1$,
\begin{equation}
    (\nabla\cL(\theta^*+\Delta))^T\frac{\Delta}{\|\Delta\|_2} =
        \frac{df(t)}{dt}\bigg|_{t=1}
    \ge \frac{df(t)}{dt}\bigg|_{t=\frac{1}{\|\Delta\|_2}}
    =(\nabla\cL(\theta^*+\Delta/\|\Delta\|_2))^T\frac{\Delta}{\|\Delta\|_2}.
\end{equation}
So
\begin{equation}
    (\nabla\cL(\theta^*+\Delta)-\nabla\cL(\theta^*))^T\Delta
        \ge(\nabla\cL(\theta^*+\Delta/\|\Delta\|_2)-\nabla\cL(\theta^*))^T\Delta.
\end{equation}
We then turn the condition~(b) to a homogeneous form
\begin{equation}
    (\nabla\cL(\theta^*+\Delta/\|\Delta\|_2)-\nabla\cL(\theta^*))^T\Delta
        \ge\alpha_2\|\Delta\|_2-\tau_2\sqrt{\frac{\log d}{n}}\|\Delta\|_1.
\end{equation}
So we can assume that $\|\Delta\|_2=1$
because it will automatically be true for $\|\Delta\|_2>1$.
Use condition~(a) when $\|\Delta\|_2=1$,
we have
\begin{equation}
    (\nabla\cL(\theta^*+\Delta)-\nabla\cL(\theta^*))^T\Delta
        \ge\alpha_1-\tau_1\frac{\log d}{n}\|\Delta\|_1^2.
\end{equation}
we only need to verify that
\begin{equation}
    \alpha_1-\tau_1\frac{\log d}{n}\|\Delta\|_1^2
        \ge\alpha_2-\sqrt{\frac{\log d}{n}}\|\Delta\|_1.
\end{equation}
This is equivalent to
\begin{equation}
    \tau_1\frac{\log d}{n}\|\Delta\|_1^2
        \le\sqrt{\frac{\log d}{n}}\|\Delta\|_1,
\end{equation}
and equivalent to
\begin{equation}
    \tau_1\sqrt{\frac{\log d}{n}}\|\Delta\|_1
        \le1,
\end{equation}
This holds because
\begin{equation}
    \tau_1\sqrt{\frac{\log d}{n}}\|\Delta\|_1
        \le\tau_1\sqrt{\frac{\log d}{4R^2\tau_1^2\log d}}2R
        =1.
\end{equation}
