\section{Problem 3}~\label{sec:prob3}

\subsection{} % 3.1

Let $f(x):=\|x\|_1$,
then the subgradient of $f$ at point $x=(x_1,x_2,\dots,x_n)$ is
\begin{equation}
\partial f(x)=\{g~|~(g_i = 1 \text{ if } x_i>0)
\text{ and } (g_i = -1  \text{ if } x_i<0)
\text{ and } (-1\le g_i \le 1  \text{ if } x_i=0)
\}.
\end{equation}

So the optimality condition for the problem is
$0\in \mu\partial f(x) + (x-v)$,
which is equivalent to the condition $(v-x)/\mu \in\partial f(x)$.
The the point $x^*=(x_1,x_2,\dots,x_n)$
which reaches the minimum should satisfy
\begin{equation}
x_i=
\begin{cases}
v_i-\mu \quad \text{ if } v_i> \mu,\\
v_i+\mu \quad \text{ if } v_i< \mu,\\
0  \quad \quad \quad \text{ if } -\mu\le v_i\le \mu.
\end{cases}
\end{equation}

\subsection{} % 3.2
