\section{Introduction} \label{sec:intro}

\subsection{Related Work}

Generative Adversarial Networks (GAN)~\cite{goodfellow2014generative}
as generative models have been actively studied
and developed~\cite{chen2016infogan,nowozin2016f,
arjovsky2017wasserstein,zhao2016energy,radford2015unsupervised,
mescheder2017adversarial,mirza2014conditional,gauthier2014conditional,
odena2016conditional,denton2015deep,reed2016generative,
huang2016stacked,zhang2016stackgan,kim2017learning,zhu2017unpaired,
che2016mode,donahue2016adversarial,salimans2016improved,zhu2016generative}
in the last few years.
There are theoretical discussions~\cite{arjovsky2017wasserstein,
zhao2016energy,nowozin2016f},
various extensions~\cite{chen2016infogan,che2016mode,donahue2016adversarial,
salimans2016improved,mescheder2017adversarial,mirza2014conditional,
gauthier2014conditional,huang2016stacked},
exploring effective network design and
training methods~\cite{radford2015unsupervised},
and applications in image generation~\cite{odena2016conditional,denton2015deep,
reed2016generative,zhang2016stackgan},
manipulation~\cite{zhu2016generative},
and cross domain transfer~\cite{kim2017learning,zhu2017unpaired}, etc.
