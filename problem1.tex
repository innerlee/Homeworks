\section{Problem 1}

Both the Frobenius norm and the nuclear norm are invariant
under left or right multiplication by a orthogonal matrix.
Thus the problem is can be rewritten as
\begin{equation}
\begin{split}
    \prox_{\|\cdot\|_*}(X)
        &= \argmin_{Z\in\real^{m\times n}}\frac{1}{2}\|X-Z\|_F^2+\|Z\|_* \\
        &= \argmin_{Z\in\real^{m\times n}}\frac{1}{2}\|U^TXV-U^TZV\|_F^2+\|U^TZV\|_* \\
        &= \argmin_{Y\in\real^{m\times n}}\frac{1}{2}\|\Sigma-Y\|_F^2+\|Y\|_*, \\
\end{split}
\end{equation}
where $Y:=U^TZV$.
We only need to prove that $Y=\Sigma_1$ is the minimum point
in the simplified formulation.
Since the problem is convex (because it is the sum of two norms),
it suffices to show that the first order optimality condition holds
at point $Y=\Sigma_1$,
\begin{equation}
    0\in \partial_Y\bigg(\frac{1}{2}\|\Sigma-Y\|_F^2+\|Y\|_*\bigg)\bigg|_{Y=\Sigma_1}.
\end{equation}

To simplify notation,
denote $\sigma_i:=\sigma_i(X), i=1,2,\dots,m$,
and suppose the first $r$ eigenvalues are greater than $1$,
\ie
\begin{equation}
    \sigma_1\ge\sigma_2\ge\cdots\ge\sigma_r>1\ge\sigma_{r+1}\ge\cdots\ge\sigma_m\ge0.
\end{equation}
Then we can write $\Sigma$ and $\Sigma_1$ as
\begin{equation}
    \Sigma=\diag(\sigma_1,\dots,\sigma_m),
\end{equation}
and
\begin{equation}
    \Sigma_1=\diag(\sigma_1-1,\dots,\sigma_r-1,0,\dots,0).
\end{equation}

Denote $W_1=\diag(\sigma_{r+1},\dots,\sigma_m)$.
Then the subgradient at $\Sigma_1$ is
\begin{equation}
\begin{split}
    & \partial_Y\bigg(\frac{1}{2}\|\Sigma-Y\|_F^2+\|Y\|_*\bigg)\bigg|_{Y=\Sigma_1} \\
    &= \bigg\{\Sigma_1 - \Sigma +
    \begin{bmatrix}
        I_r  & 0 \\
        0    & W \\
    \end{bmatrix}: \|W\|\le 1 \bigg\} \\
    &= \bigg\{\begin{bmatrix}
        -I_r  & 0 \\
        0    & -W_1 \\
    \end{bmatrix} +
    \begin{bmatrix}
        I_r  & 0 \\
        0    & W \\
    \end{bmatrix}: \|W\|\le 1 \bigg\} \\
    &= \bigg\{\begin{bmatrix}
        0    & 0 \\
        0    & W - W_1 \\
    \end{bmatrix}: \|W\|\le 1 \bigg\} \\
\end{split}
\end{equation}
where $\|W\|$ is the spectral norm,
and we used the result from HW2, Problem 2 to compute the subgradient
of nuclear norm.
Since $\|W_1\|=\sigma_{r+1}\le 1$,
by simply making $W=W_1$,
we know $0$ belongs to the subgradient of the objective,
as desired.
